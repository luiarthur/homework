\documentclass[mathserif, 11pt, t]{beamer}

\let\Tiny=\tiny
\usepackage{xcolor}
\usepackage{pdfpages}
\usepackage{graphicx}
\usepackage{float}
\usepackage{enumitem}
\usepackage{natbib}
\usepackage{amsmath}
\usepackage{bm}
\geometry{vmargin=0.5in}

%colors
\definecolor{blue}{rgb}{0.05, 0.05, 0.90}
\definecolor{bluegreen}{rgb}{0.05, 0.2, 0.6}

%commands
\newcommand{\E}{\mathrm{E}}
\newcommand{\Var}{\mathrm{Var}}
\newcommand{\lra}{\longrightarrow}
\newcommand{\ra}{\rightarrow}
\newcommand{\m}[1]{\mathbf{\bm{#1}}}
\renewcommand{\subtitle}[1]{\vspace{0.45cm}\textcolor{bluegreen}{
    {\textbf{#1}}}\vspace{0.15cm}\newline}
\newcommand{\tlb}[1]{\large{\textbf{#1}}}

%slide colors
\pagecolor{blue!70}

\begin{document}

%%% begin title frame
\begin{center}
\ \\ [-0.5in]
\vfill
\bigskip
\bigskip
\bigskip
\bigskip
\bigskip

\begin{LARGE}
\begin{center}
GLMM Parameter Estimation via Approximation Methods
\end{center}
\end{LARGE}
\vfill

\begin{center}
Mickey Warner
\end{center}
\vfill
April 2015
\bigskip
\bigskip
\bigskip
\vfill
\ \\ [-0.5in]
\end{center}
%%% end title frame

\begin{frame}
\subtitle{}
This is a review of ``Approximate Inference in Generalized Linear Mixed Models'' by Breslow and Clayton in 1993.

\end{frame}

\begin{frame}
\subtitle{Hierarchical model}
The $i$th of $n$ observations has univariate response $y_i$ with vectors $\m{x}_i$ and $\m{z}_i$ is covariates.
\bigskip

We have a $q$-vector $\m{b}$ of random effects, where $\m{b}\sim N(\m{0},\m{D}(\m{\theta}))$.
\bigskip

In vector notation, the conditional mean of $\m{y}=(y_1,\ldots,y_n)^\top$ given $\m{b}$ is assumed to satisfy
\[ \E(\m{y}|\m{b}) = \m{\mu} = h(\m{X}\m{\alpha} + \m{Z}\m{b}) \]

and $\Var(y_i|\m{b})=\phi a_iv(\mu_i)$ for $i=1,\ldots,n$, for known constant $a_i$ and variance function $v(\cdot)$. Note, we do not have any distributional assumption on $y_i$.

\end{frame}

\begin{frame}
\subtitle{Quasi-likelihood function}

For independent observations $y_i,\ldots,y_n$, the quasi-likelihood function is defined as
\begin{eqnarray*}
\frac{\partial K(y_i, \mu_i)}{\partial \mu_i} &=& \frac{y_i-\mu_i}{a_i v(\mu_i)} \\
\Longrightarrow K(y_i,\mu_i)&=&\int_{y_i}^{\mu_i}\frac{y_i-u_i}{a_i v(u_i)}du_i
\end{eqnarray*}

This has properties similar to the log-likelihood of a distribution from the exponential family.


\end{frame}

\begin{frame}

\bigskip
\bigskip

The integrated quasi-likelihood function is defined by
\begin{eqnarray*}
e^{ql(\m{\alpha}, \m{\theta})} &\propto& |\m{D}|^{-1/2}\int\exp\left[\frac{1}{\phi}\sum_{i=1}^nK(y_i,\mu_i)-\frac{1}{2}\m{b}^\top\m{D}^{-1}\m{b} \right]d\m{b} \\
&=& c|\m{D}|^{-1/2}\int\exp\left[ \kappa(\m{b}) \right] d\m{b} 
\end{eqnarray*}
where $c$ is some multiplicative constant. Using Laplace's method for integral approximation (and ignoring the constants), we have
\[ql(\m{\alpha}, \m{\theta}) \approx -\frac{1}{2}\log|\m{D}| -\frac{1}{2}\log|\kappa''(\tilde{\m{b}})| - \kappa(\tilde{\m{b}}) \]
where $\tilde{\m{b}}$ is the solution to $\kappa'(\m{b})=0$.

\end{frame}

\begin{frame}
\subtitle{Penalized quasi-likelihood (PQL)}

\subtitle{Marginal quasi-likelihood (MQL)}

\end{frame}



\begin{frame}
\subtitle{Estimating parameters}

\end{frame}



\begin{frame}
\subtitle{Roughbutts and Whitesides}

\end{frame}

% \begin{itemize}[label={$\cdot$}]
% \item I
% \item Item 2
% \end{itemize}

\end{document}
