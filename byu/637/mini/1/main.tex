\documentclass[12pt]{article}

\usepackage{amsmath}
\usepackage{bm}

\newcommand{\m}[1]{\mathbf{\bm{#1}}}
\newcommand{\R}{I\hspace{-4.4pt}R}

\newcommand{\E}{\mathrm{E}}
\newcommand{\var}{\mathrm{var}}

% Brief summary of the paper (highlight important parts as you
% would in a literature review, where this paper is a main source
% for your work -- not everything, just important details).

% Comment on how the paper applies to the class (ideas for each
% paper are listed with each citation, although you are not limited
% to these suggestions).

% Extend the results of the paper to yoru own analysis.


\begin{document}

\noindent Mickey Warner
\bigskip

\noindent Stat 637 -- Mini Project \# 1

\section*{Introduction}

\noindent In this report, we review Zeger \emph{et al.}'s 1988 paper titled \emph{Models for Longitudinal Data: A Generalized Estimating Equation Approach}. The authors consider two classes of models: subject-specific (SS) and population average (PA). These models are employed when several measurements are collected on individuals across time. These measurements are typically correlated which adds an additional challenge in the analysis.
\bigskip

\noindent Subject-specific models are used when the response for an individual is the focus. These models taken on the form of a generalized linear mixed model. For subject $i$ at time $t$, let $y_{it}$ be the response, $\m{x}_{it}$ a $p\times 1$ vector of fixed covariates associated with $p\times 1$ fixed effects $\m{\beta}$, and $\m{z}_{it}$ a $q\times 1$ vector of covariates associated with $q\times 1$ random effects $\m{b}_i$. Let $u_{it}=E(y_{it}|\m{b}_i)$. We assume the responses satisfy
\[ h(u_{it}) = \m{x}_{it}^\top\m{\beta} + \m{z}_{it}^\top\m{b}_i ~~~\mathrm{and}~~~ \var(y_{it}|\m{b}_i) = g(u_{it})\cdot \phi \]
where $\m{b}_i$ is an independent observation from some distribution, $F$, and $i=1,\ldots,K$ and $t=1,\ldots,n_i$. Typically, $\m{b}_i\sim N(\m{0}, \m{D})$. The functions $h$ and $g$ are referred to as the ``link'' and ``variance'' functions, respectively. Choices for $h$ include the log, logit, or probit links. The variance function $g$ may be defined by the choice of likehood (e.g. with a Poisson likelihood, we have $\mathrm{var}(y_{it}|\m{b}_{it})=u_{it}$). The scale parameter $\phi$ can be used to define quasi-likelihoods.
\bigskip

\noindent When inference on a population is of more interest than inference on a subject, an alternative model is the population average model. Let $\mu_{it}=E(y_{it})$ be the marginal expectation. The responses then satisfy
\[ h^*&(u_{it}) = \m{x}_{it}^\top\m{\beta}^* ~~~\mathrm{and}~~~ \var(y_{it}) = g^*(u_{it})\cdot \phi, \]
for link and variance functions $h^*$ and $g^*$.


\end{document}
